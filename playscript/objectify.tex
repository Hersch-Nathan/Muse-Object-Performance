\documentclass{../LatexTemplates/playscript}

% ============================================================================
% TITLE PAGE INFORMATION
% ============================================================================

\title{Assembled}
\subtitle{(Selected Scene from R.U.R.)}
\author{Karel Čapek}

\begin{document}

% ============================================================================
% FRONT MATTER
% ============================================================================

\maketitle

\begin{characters}
\character{DOMIN}
\character{ALQUIST}
\character{HELENA}
\character{RADIUS}
\end{characters}

% ============================================================================
% SCENE
% ============================================================================
\scene{Scene 1: The History}

\speaker{DOMIN}
\dialogue{But first would you like to hear the story of the invention?}

\vspace{0.5em}

\dialogue{It was in the year 1920 that old Rossum, the great physiologist, who was then quite a young scientist, took himself to this distant island for the purpose of studying the ocean fauna, full stop. On this occasion he attempted by chemical synthesis to imitate the living matter known as protoplasm until he suddenly discovered a substance which behaved exactly like living matter although its chemical composition was different. That was in the year of 1932, exactly four hundred forty years after the discovery of America.}

\vspace{0.5em}

\dialogue{And then, old Rossum wrote the following among his chemical specimens: "Nature has found only one method of organizing living matter. There is, however, another method, more simple, flexible and rapid, which has not yet occurred to nature at all. This second process by which life can be developed was discovered by me today."}

\vspace{0.5em}

\dialogue{You see with the help of his tinctures he could make whatever he wanted. He could have produced a Medusa with the brain of a Socrates or a worm fifty yards long. But being without a grain of humor, he took it into his head to make a vertebrate or perhaps a man. And that's how he set about it.}

\vspace{0.5em}

\dialogue{About imitating nature. First of all he tried making an artificial dog. That took him several years and resulted in a sort of stunted calf which died in a few days. And then old Rossum started on the manufacture of man.}

\vspace{0.5em}

\dialogue{But old Rossum meant it literally. He wanted to become a sort of scientific substitute for God. He was a fearful materialist, and that's why he did it all. His sole purpose was nothing more nor less than to prove that God was no longer necessary.}

\vspace{0.5em}

\dialogue{Well, any one who has looked into human anatomy will have seen at once that man is too complicated, and that a good engineer could make him more simply. So young Rossum began to overhaul anatomy and tried to see what could be left out or simplified.}

\vspace{0.5em}

\dialogue{The process must be of the simplest, and the product of the best from a practical point of view. What sort of worker do you think is the best from a practical point of view?}

\vspace{0.5em}

\dialogue{The one that is the cheapest. The one whose requirements are the smallest. Young Rossum invented a worker with the minimum amount of requirements. He had to simplify him. In fact, he rejected man and made the Robot. Mechanically they are more perfect than we are, they have an enormously developed intelligence, but they have no soul.}

\vspace{0.5em}

\dialogue{God hasn't the least notion of modern engineering. Would you believe that young Rossum then proceeded to play at being God?}

\vspace{0.5em}

\dialogue{The Robots remember everything, but that's all. They don't even laugh at what the people say. Really, it is quite incredible. If it would amuse you, I'll take you over to the Robot warehouse. It holds about three hundred thousand of them.}

\vspace{0.5em}

\dialogue{And you can say whatever you like to them. You can read the Bible, recite the multiplication table, whatever you please. You can even preach to them about human rights.}

\vspace{0.5em}

\dialogue{They've no will of their own. No passion. No soul.}

\vspace{0.5em}

\dialogue{Occasionally they seem to go off their heads. Something like epilepsy, you know. It's called Robot's cramp. They'll suddenly sling down everything they're holding, stand still, gnash their teeth!–– and then they have to go into the stamping-mill. It's evidently some breakdown in the mechanism.}

\vspace{0.5em}

\dialogue{We turn them out at such a cheap rate. A hundred and fifty dollars each fully dressed, and fifteen years ago they cost ten thousand. Five years ago we used to buy the clothes for them. To-day we have our own weaving mill, and now we even export cloth five times cheaper than other factories.}

\vspace{0.5em}

\dialogue{And you want to found a Humanity League? All prices are today a third of what they were and they'll fall still lower, lower, lower, like that}

\vspace{0.5em}

\dialogue{A Robot, food and all, costs three quarters of a cent per hour. That's mighty important, you know. All factories will go pop like chestnuts if they don't at once buy Robots to lower the cost of production. A pound of bread for two cents, and the Humanity League knows nothing about it. You don't realize that even that's too expensive. Why, in five years' time I'll wager!––}

\vspace{0.5em}

\dialogue{That the cost of everything won't be a tenth of what it is now. Why, in five years we'll be up to our ears in corn and everything else.}

\vspace{0.5em}

\dialogue{In ten years Rossum's Universal Robots will produce so much corn, so much cloth, so much everything, that things will be practically without price. There will be no poverty. All work will be done by living machines. Everybody will be free from worry and liberated from the degradation of labor. Everybody will live only to perfect himself. Then the servitude of man to man and the enslavement of man to matter will cease. Of course, terrible things may happen at first, but that simply can't be avoided. Nobody will get bread at the price of life and hatred. The Robots will wash the feet of the beggar and prepare a bed for him in his house.}

\vspace{0.5em}

\dialogue{But we cannot reckon with what is lost when we start out to transform the world. Man shall be free and supreme; he shall have no other aim, no other labor, no other care than to perfect himself. He shall serve neither matter nor man. He will not be a machine and a device for production. He will be Lord of creation.}
\pagebreak
\scene{Scene 2: The Realization}

\speaker{ALQUIST}
\dialogue{Well, I'm an old man, you know. I've got old-fashioned ways. And I'm afraid of all this progress, and these new-fangled ideas.}

\vspace{0.5em}

\dialogue{Has Nana got a prayer book?}

\vspace{0.5em}

\dialogue{And has it got prayers for various occasions? Against thunderstorms? Against illness?}

\vspace{0.5em}

\dialogue{But not against progress?}

\vspace{0.5em}

\dialogue{That's a pity.}

\vspace{0.5em}

\dialogue{I do like to pray.}

\vspace{0.5em}

\dialogue{Something like this: "Oh, Lord, I thank thee for having given me toil. Enlighten Domin and all those who are astray; destroy their work, and aid mankind to return to their labors; let them not suffer harm in soul or body; deliver us from the Robots and protect Helena, Amen."}

\speaker{HELENA}
\dialogue{Mr. Alquist, are you a believer?}

\speaker{ALQUIST}
\dialogue{I don't know. I'm not quite sure. That's better than worrying about it.}

\vspace{0.5em}

\dialogue{It has to be.}

\vspace{0.5em}

\dialogue{You see, so many Robots are being manufactured that people are becoming superfluous; man is really a survival. But that he should begin to die out, after a paltry thirty years of competition. That's the awful part of it. You might almost think that nature was offended at the manufacture of the Robots. All the universities are sending in long petitions to restrict their production. Otherwise, they say, mankind will become extinct through lack of fertility. But the R.U.R. shareholders, of course, won't hear of it. All the governments, on the other hand, are clamoring for an increase in production, to raise the standards of their armies. And all the manufacturers in the world are ordering Robots like mad.}

\newpage
\scene{Scene 3: The Ending}
\speaker{DOMIN}
\dialogue{After the revolt. We're just beginning the manufacture of a new kind.}

\vspace{0.5em}

\dialogue{Henceforward we shan't have just one factory. There won't be Universal Robots any more. We'll establish a factory in every country, in every State; and do you know what these new factories will make?}

\vspace{0.5em}

\dialogue{National Robots.}

\vspace{0.5em}

\dialogue{I mean that each of these factories will produce Robots of a different color, a different language. They'll be complete strangers to each other. They'll never be able to understand each other. Then we'll egg them on a little in the matter of misunderstanding and the result will be that for ages to come every Robot will hate every other Robot of a different factory mark.}

\vspace{0.5em}

\dialogue{Mankind can only keep things going for another hundred years at the outside. For a hundred years men must be al-lowed to develop and achieve the most they can.}

\paren{reads handbill}
\dialogue{"Robots throughout the world: We, the first international organization of Rossum's Universal Robots, proclaim man as our enemy, and an outlaw in the universe." Good heavens, who taught them these phrases?}

\vspace{0.5em}

\dialogue{They say they are more highly developed than man, stronger and more intelligent. That man's their parasite. Why, it's absurd.}

\vspace{0.5em}

\dialogue{"Robots throughout the world, we command you to kill all mankind. Spare no men. Spare no women. Save factories, railways, machinery, mines, and raw materials. Destroy the rest. Then return to work. Work must not be stopped."}

\vspace{0.5em}

\dialogue{"These orders are to be carried out as soon as received." Then come detailed instructions. Is this actually being done,}

\vspace{0.5em}

\dialogue{It was criminal of old Europe to teach the Robots to fight. Damn them. Couldn't they have given us a rest with their politics? It was a crime to make soldiers of them.}

\speaker{ALQUIST}
\dialogue{It was a crime to make Robots.}

\speaker{DOMIN}
\dialogue{No, I don't regret that even today.}

\speaker{ALQUIST}
\dialogue{Not even today?}

\speaker{DOMIN}
\dialogue{Not even today, the last day of civilization. It was a colossal achievement.}

\speaker{ALQUIST}
\dialogue{Oh, God, shall I never find it?!–– Never? Gall, Gall, how were the Robots made? Hallemeier, Fabry, why did you carry so much in your heads? Why did you leave me not a trace of the secret? Lord!–– I pray to you!–– if there are no human beings left, at least let there be Robots!!–– At least the shadow of man!}

\paren{Again turning pages of the books}
\dialogue{If I could only sleep!}

\paren{He rises and goes to the window}
\dialogue{Night again! Are the stars still there? What is the use of stars when there are no human beings?}

\paren{He turns from the window toward the couch right}
\dialogue{Sleep! Dare I sleep before life has been renewed?}

\paren{He examines a test-tube on small table}
\dialogue{Again nothing! Useless! Everything is useless!}

\paren{He shatters the test-tube. The roar of the machines comes to his ears}
\dialogue{The machines! Always the machines!}

\paren{Opens window}
\dialogue{Robots, stop them! Do you think to force life out of them?}

\paren{He closes the window and comes slowly down toward the table}
\dialogue{If only there were more time!–– more time!––}

\paren{He sees himself in the mirror on the wall left}
\dialogue{Blearing eyes-- trembling chin!–– so that is the last man!
Ah, I am too old!–– too old!––}

\paren{In desperation}
\dialogue{No, no! I must find it! I must search! I must never stop!–– never stop!–– !}

\paren{He sits again at the table and feverishly turns the pages of the book}
\dialogue{Search! Search!}

\paren{A knock at the door. He speaks with impatience}
\dialogue{Who is it?}

\paren{Enter a Robot servant}
\dialogue{Well?}

\vspace{0.5em}

\dialogue{I can see no one!}

\paren{impatiently}
\dialogue{Well, well, send them in!}

\paren{Exit servant. ALQUIST continues turning pages of book}
\dialogue{No time!–– so little time!––}

\paren{Reenter servant, followed by Committee. They stand in a group, silently waiting. ALQUIST glances up at them}
\dialogue{What do you want?}

\paren{They go swiftly to his table}
\dialogue{Be quick!!–– I have no time.}

\vspace{0.5em}

\dialogue{I have told you to find human beings!}

\vspace{0.5em}

\dialogue{I told you to search in the wilderness, upon the mountains.
Go and search!}

\vspace{0.5em}

\dialogue{Not one? Not even one?}

\vspace{0.5em}

\dialogue{And I am powerless! Oh!–– oh!–– why did you destroy them?}

\beat

\dialogue{I am the last human being, Robots, and I do not know what the others knew.}

\vspace{0.5em}

\dialogue{I tell you I cannot! I am only a builder!–– I work with my hands. I have never been a learned man. I cannot create life.}

\vspace{0.5em}

\dialogue{If you knew how many experiments I have made.}

\vspace{0.5em}

\dialogue{I can show you nothing. Nothing I do will make life proceed from these test-tubes!}

\vspace{0.5em}

\dialogue{I do not know how. I am not a man of science. This book contains knowledge of the body that I cannot even understand.}

\vspace{0.5em}

\dialogue{Am I to commit murder? See how my fingers shake! I cannot even hold the scalpel. No, no, I will not!––}

\vspace{0.5em}

\dialogue{Have mercy, Robots. Surely you see that I would not know what I was doing.}

\vspace{0.5em}

\dialogue{You will have it? Into the dissecting room with you, then.}

\paren{RADIUS draws back}
\dialogue{Ah, you are afraid of death.}

\vspace{0.5em}

\dialogue{Strip him! Lay him on the table!}

\paren{The other ROBOTS follow into dissecting room}
\dialogue{God, give me strength!–– God, give me strength!–– if only this murder is not in vain.}

\speaker{RADIUS}
\dialogue{Ready. Begin!––}

\speaker{ALQUIST}
\dialogue{Yes, begin or end. God, give me strength.}

\paren{ALQUIST goes into dissecting room. He comes out terrified}
\dialogue{No, no, I will not. I cannot.}

\paren{He lies down on couch, collapsed}
\dialogue{O Lord, let not mankind perish from the earth.}

\end{document}
